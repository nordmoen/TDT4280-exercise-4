\section{Auction Type}\label{sec:auction}

Our chiose of auction type is the recommended First-price Sealed-Did (FPSB).
Mostly because it's eeasy to implement. The ease of implementation gave us more
time for debugging, and eased the rest of the rest of the work on the exercise. 

FPSB's strenghts are ease of implementation and potential speed. The auction uses little
time to aquire bids. And we don't waste time calculating best scores or changing
parameters. 

FPSB aquires bids by asking all agents for bids. The auction waits for replies.
And then the bids are compared. Then reject and accept messages are sent to all
the agents, where only one gets an accept message. The agent with the favored
bid is then sent the task to execute. 

A possible problem is the time the auction has
to wait for replies. The auction uses \textit{recieve-blocking}, which waits
for a reply before continuing. If the agent doesn't reply quickly, or something
goes wrong in the process, the waiting time might be long. Althought the possibility of this is
slim in the test environment.

Other possible auction techniques are: 
A range of iterative auctions, All-pay auctions and Reserve auction. 

In an all-pay auction everyone must pay their bid, even if they don't win the
auction. This form for bidding makes the bidder committ to the auction, and
possibly stay longer because of the already paid stakes. 

The English auction as an alternative to our choise would slow down the
program. As the English auction tries to find a price which only one bidder is
willing to pay. This might result in many bidding rounds.

Reserve auctions aim to aquire the lowest price from a seller. The buyers ask
for the lowest price from multiple sellers, and buys from the seller with the
lowest price. Here the sellers have a reserved price which is their lower
limit. This limit might not be known to the buyer or the other sellers. The
seller can also choose to not sell if the asked price is lower than the
reserve limit. 

