\section{Decompose Example}\label{sec:decompose-example}
As an example we can show some of the operations which go into solving something
like the equation in listing \ref{code:example-operation}.

\begin{lstlisting}[frame=single, breaklines=true, caption=Example equation,
label=code:example-operation]
+ 10 * 2 / 10 - 3 1
\end{lstlisting}

The first thing that happens when the TaskAdmin receives the equation is that it
is parsed into a more appropriate representation, which is accomplished through
the MathProblem class. This is then appended to a list of problems to be solved
and the TaskAdmin is then asked to solve the first problem in that list. We are
now soon starting to solve the problem with communication. The next thing that
happens is that the problem is divided up into its smaller parts, recursively,
once this methods has found a leaf calculation it will start to solve this using
auction.

Once a leaf calculation is reached that calculation is set to be solved. The
TaskAdmin checks which operation which is needed to solve it, e.g. minus, plus,
etc. It then contacts the directory service agent and asks it for all the agents
which has registered it self as solver for the operation it is looking for. The
directory agent will return a list of agents and the TaskAdmin will then start
the auction(this is described in section \ref{sec:auction}). When a solver is
chosen the TaskAdmin will contact all the ones that lost and send them a
"Reject-proposal" message telling them that they did not win the auction. The
TaskAdmin will then contact the winner and send it a "Accept-proposal" and then
wait for the result.

The leaf calculation is then done and the result is sent back up to the
recursive function which then either combine that result into a more complex
equation or it will return this as the result replying back to the agent which
asked it in the first place.

We have attached a text file together with this report which displays the actual
messages sent to solve the equation in \ref{code:example-operation}.
