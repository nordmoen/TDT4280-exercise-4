\section{Implementation}\label{sec:implementation}
The implementation we have written is centered around two central classes. The
first class is the Task administrator, located in \textit{TaskAdmin.java}, which
is quite straight forward and just contains code for facilitating auctions and
receiving tasks. All of which is quite according to the assignment given and
does not deviate much. The code uses static blocking receives and not any of the
built-in behaviors. The main methods are \textit{solve} and \textit{auction}
which together solves each problem given. The solve method just recursively
solves the main problems by decomposing it down in to the leaf calculations and
then joining those results and then solving those calculation. Each calculation
is auctioned to corresponding solvers using the auction method. The
auction method is sequential in nature, but the Task administrator as a whole
could be made to solve problems parallel quite easily by making it an thread and
changing the list of problems to a concurrent data structure.

The second central class is the \textit{AbstractSolver.java} which is the base
class of all solvers. This class does all the base work and only leaves the
calculation and time estimation to subclasses. Because this class does not need
any recursive behavior the general behavior is inherently asynchronous. This is
done by having all messages and all interactions end by returning an answer. The
receiving of messages is done through a cyclic behaviour which waits for
messages and then acts on them using different methods. Because of this
asynchronism the class stores each bid in a list and retrieves them when getting
a follow up message before it acts on that stored bid, either solving it or
discarding the bid altogether.
